\newcommand\thesistitle{
    Computational models for research in Medicine and Desalination 
}
\newcommand\nameanddegrees{
    Andrew Philip Freiburger\\
    A.S., Grand Rapids Community College, 2016\\
    B.S., Grand Valley State University, 2019\\
    \tpbreak
}
\newcommand\panel{
    \HRule\\\panelist{Dr. Heather L. Buckley}{Supervisor}{Department of Civil Engineering}
    \HRule\\\panelist{Dr. Irina Paci}{Outside Member}{Department of Chemistry}
}

% Titlepage
\newcommand\tpbreak{\\[\baselineskip]}
\newpage
\thispagestyle{empty}

% Header Footer Setup (for Front matter)
\pagestyle{myheadings}
\pagenumbering{roman}
% We use this in addition to the default \LaTeX page configuration routines
% because we have no way of saying \thispagestyle after the glossary and bibliography starts.
\fancypagestyle{plain}{%
    \fancyhf{}
    \fancyhead[R]{\thepage}
    \renewcommand{\headrulewidth}{0pt}
    \renewcommand{\footrulewidth}{0pt}
}

%%%%%%%%%%%%%%%%%%%%%%%%%%%% FRONT MATTER %%%%%%%%%%%%%%%%%%%%%%%%%%%%%%%%%
\pagebreak
{
    \centering
    \thesistitle
    \tpbreak
    
    by
    \tpbreak
    \nameanddegrees
    A Thesis Submitted in Partial Fulfillment of the \\
    Requirements for the Degree of
    \tpbreak
    MASTER OF APPLIED SCIENCE
    \tpbreak
    in the Department of Civil Engineering\\
    \vfill
    \begin{tabular}{cl}
        & \copyright\ Andrew Philip Freiburger 2022\\
        & \phantom{\copyright} University of Victoria
    \end{tabular}
    \tpbreak
    All rights reserved. This Thesis may not be reproduced in whole or in part, by \\
    \hfill photocopying or other means, without the permission of Andrew Philip Freiburger. 
    \hfill
}
\pagebreak

\newpage
\TOCadd{Supervisory Committee}
{
    \centering
    \thesistitle
    \tpbreak
    by
    \tpbreak
    \nameanddegrees
}
\newcommand\panelist[3]{\noindent #1, #2\\\noindent(#3)\tpbreak}
\vfill
\noindent Supervisory Committee
\tpbreak
\panel
\vfill
\pagebreak

\newpage
\TOCadd{Abstract}


% Abstract
\begin{center}
    \textbf{ABSTRACT}
\end{center}
\begin{justify}
    The development of sustainable and practical technologies is essential for the continuation of civilization. Two problems that are particularly imperative for society to resolve are 1) water insecurity and 2) antimicrobial resistance. Water insecurity may be alleviated with desalination technologies, however, desalination is prone to a membrane fouling that hinders its practicality for low-resource contexts. The two primary types of membrane fouling are scaling -- mineral precipitation and deposition upon the membrane -- and biofouling -- microbial colonization of the polymeric filtration membrane. The treatment of biofouling with antibiotics is intertwined with the antimicrobial resistance (AMR) crisis, where AMR infections are projected to exceed cancer in annual deaths by the mid-21$\text{st}$ century. The AMR crisis may be mitigated through photodynamic inactivation (PDI), which uses reactive oxygen species (ROSs) to non-selectively oxidize and kill pathogens sufficiently fast to avoid adaptive mechanisms that result in AMR. The innumerable possible combinations of control and experimental variables in studies of membrane fouling and PDI are unlikely to be completely explored experimentally, where resource limitations restrain experimentation. This Thesis, therefore, developed models and Python application programming interfaces (APIs) that can 1) explore continuums of parameter values and 2) predict the efficacy of desalination or PDI systems. These open-source Python modules may expedite the development of practical technologies that resolve water insecurity and stymie antibiotic resistant epidemics, thereby improving the likelihood of a long-lived civilization far into the future. 
\end{justify}

\TOCadd{Table of Contents}\tableofcontents
\TOCadd{List of Tables}\listoftables
\setcounter{lofdepth}{2}
\TOCadd{List of Figures}\listoffigures
\input frontmatter/ack

% Header Footer Setup
\newpage
\pagestyle{myheadings}
\pagenumbering{arabic}
% We use this in addition to the default LaTeX page configuration routines
% because we have no way of saying \thispagestyle after the bibliography starts.
\fancypagestyle{plain}{%
    \fancyhf{}
    \fancyhead[R]{\ifnum\thepage=1\relax\else\thepage\fi}
    \renewcommand{\headrulewidth}{0pt}
    \renewcommand{\footrulewidth}{0pt}
}
