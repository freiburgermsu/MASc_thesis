\startchapter{The New Approach and Solution}
\label{chapter:newsol}

This is where you go all out and tell us all about your new discovery and research related to the problem in the previous chapter. No arrogant sweeping statements which cannot be fully justified, but no false modesty either. You must impress your reader that you have accomplished something.

Simply summarized, this chapter should be comprised of at least two main sections, each with appropriate subsections. The first section should describe:
\begin{itemize}
\item {what the new approach is;}
\item {what is really totally new;}
\item {what is incrementally new;}
\item {what you built upon.}
\end{itemize}

The second part should describe fully how the new approach works, both with the overall theoretical exposition (e.g. an algorithm) and with as many examples as necessary for clarity. Remember that if the reader does not understand fully, you will get a lot of questions and doubts. Good examples, good figures, good diagrams with super clear tutorial explanations can be a joy to read and make even a small contribution appear to be more impressive. Are you afraid that if you are too tutorial your work will not seem as deep and difficult? Only shallow people will make such a superficial evaluation, have trust instead in the wisdom of your supervisory committee.

Use at least one good example throughout, and even better if this is one of the examples you used in Chapter 2 to describe the original problem.

By the way, this would be the first chapter I would write. This is what I know best right now, as I just finished working on it. It is clear to me and on the tip of my fingers. Start with your strengths! The second chapter I would write is the next one about the experiments, followed closely by chapter 2 describing the problem. It may not seem intuitive to you, but it works and it is the most productive way I ever found to finish a document.


\input chapters/3/sec_latexhelp
